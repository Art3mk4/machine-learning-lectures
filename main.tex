\documentclass[12pt]{article}

\usepackage[T1,T2A]{fontenc}
\usepackage[utf8]{inputenc}
\usepackage[english,russian]{babel}
\usepackage{hyperref}

\newtheorem{definition}{Определение}

\title{Лекции по машинному обучению}
\author{Игорь Чирков}
\date{2020 г.}

\begin{document}
    \begin{titlepage}
        \maketitle
    \end{titlepage}

    \begin{abstract}
        Лекции по машинному обучению, прочитанные в Luxoft
    \end{abstract}

    \section{Введение в нейросети}\label{sec:введение-в-нейросети}

    % TODO: встаивть определния из теории вероятности

    Определим круг задач.
    Наша задача - задача классификации.
    Определение задачи классификации взято из
    \href{https://en.wikipedia.org/wiki/Statistical_classification}{Википедии}.

    \begin{definition}[Задача классификации]
        Пусть дано множество $X$ $n$-мерных линейных векторов $X \ni \overline{x} = (x_1, \ldots, x_n)$,
        $Y$~--- некоторое множество классов.
        Существует неизвестная целевая зависимость~--- отображение $y^*: X\to Y$, значение которой известны только
        на объектах конечной обучающей выборки $X^m \subset X \times Y$, $| X^m | = m$.

        Требуется построить алгоритм $\alpha: X \to Y$, способный классифицировать произвольный объект
        $x \in X$
    \end{definition}

    Несложно вывести, что достаточно решить задачу классификации только на два класса.

    % TODO: вставить красивые графики классификации точек на плоскости

    % TODO: Лекция 2

    % TODO: Лекция 3
\end{document}
